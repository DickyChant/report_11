\documentclass{ctexart}
    \usepackage{mathrsfs}
    \usepackage{multirow}
    \usepackage{graphicx}
    \usepackage{array}
    \usepackage{makecell}
    \usepackage{amsmath}
    \usepackage{booktabs}
    \usepackage{float}
    \usepackage{diagbox}
    \newcommand\mgape[1]{\gape{$\vcenter{\hbox{#1}}$}}
    \newcommand\Ronum[1]{\uppercase\expandafter{\romannumeral #1\relax}}
    \newcommand\ronum[1]{\romannumeral #1\relax}
    \author{钱思天\ 1600011388 No.8}
    \title{实验十四\ 直流电桥测量电阻 \ 实验报告}
    \begin{document}
      \maketitle
      \section{实验数据与处理}
      \subsection{平衡电桥测量结果}
      % Table generated by Excel2LaTeX from sheet 'Sheet1'
% Table generated by Excel2LaTeX from sheet 'Sheet1'
\begin{table}[H]
    \centering
    \caption{不同$R_x$不同$R_1/R_2$(均$E=4.0V$ \& $R_h=0\Omega$)测量结果}
    \resizebox{\textwidth}{!}  
    {
    \begin{tabular}{|c|c|c|c|c|c|c|c|}
        \hline
      \multicolumn{2}{|l|}{\diagbox[dir=NW]{$R_x\  \&\  \frac{R_1}{R_2} $}{测量值}{各待测项}} & $R_0(\Omega )$ & $R_0'(\Omega )$ & $\Delta n$(格)& $R_x(\Omega )$ & $\Delta R_0(\Omega )$ & S \\
      \hline
      $R_{x1}$ & 500/500 & 47.9  & 47.8  & 4.0   & 47.9  & 0.1   & $1.9 \times 10^3$ \\
      \hline
      \multirow{3}[0]{*}{$R_{x2}$} & 50/500 & 3600  & 3575  & 4.0   & 360.0 & 25    & $5.8 \times 10^2$ \\
      \cline{2-8}
      & 500/500 & 360.0 & 361.0 & 4.0   & 360.0 & 1.0   & $1.4 \times 10^3$ \\
      \cline{2-8}    
      & 500/500(交换) & 360.0 & 361.0 & 4.0   & 360.0 & 1.0   & $1.4 \times 10^3$ \\
      \hline
      $R_{x3}$ & 500/500 & 4059  & 4005  & 4.0   & 4059.0 & 54    & $3.0\times 10^2$ \\
      \hline  
    \end{tabular}%
    }
    \label{tab:addlabel}%
  \end{table}%
  % Table generated by Excel2LaTeX from sheet 'Sheet1'
\begin{table}[H]
    \centering
    \caption{$R_{x2}$不同测量条件测量结果}
    \resizebox{\textwidth}{!}
    {
      \begin{tabular}{|m{0.365\columnwidth}|c|c|c|c|c|c|c|}
        \hline
        \diagbox[dir=NW,width=0.4\columnwidth]{各测量条件}{测量值}{各待测项}    & $R_0(\Omega )$ & $R_0'(\Omega )$ & $\Delta n$(格) & $R_x(\Omega )$ & $\Delta R_0(\Omega )$ & S \\ \hline
      $E=4.0V$ \& $R_h=0\Omega$ \& $R_1/R_2=500/500$ & 360.0 & 361.0 & 4.0   & 360.0 & 1.0   & $1.4 \times 10^3$ \\ \hline
      $E=2.0V$ \& $R_h=0\Omega$ \& $R_1/R_2=500/500$ & 360.0 & 362.0 & 4.0   & 360.0 & 2.0   & $7.2 \times 10^2$ \\ \hline
      $E=4.0V$ \& $R_h=0\Omega$ \& $R_1/R_2=500/5000$ & 3600  & 3650  & 4.0   & 360.0 & 50.0  & $2.9 \times 10^2$ \\ \hline
      $E=4.0V$ \& $R_h=3.0k\Omega$ \& $R_1/R_2=500/500$ & 360   & 340   & 5.5   & 360.0 & 10.0  & $2.0 \times 10^2$ \\ \hline
      \end{tabular}%
    }
    \label{tab:addlabel}%
  \end{table}%
  
  
      
 关于灵敏度$S$的计算,利用公式
 $$S=\frac{\Delta n}{\Delta R_x /R_x}=\frac{\Delta n}{\Delta R_0 /R_0}$$
 
 可计算出各$S$的实测值,已附于数据表内。

 至于$S$的理论值,根据公式$$S=\frac{S_GE}{R_1+R_2+R_3+R_4+(R_g+R_h)(2+\frac{R_1}{R_x}+\frac{R_0}{R_2})}$$

 将$S_G^{-1}=1.3\times 10^{-6}(A/\mbox{格})$及$R_g=47\Omega$代入,得下二表:
 % Table generated by Excel2LaTeX from sheet 'Sheet1'
\begin{table}[H]
  \centering
  \caption{不同$R_x$不同$R_1/R_2$(均$E=4.0V$ \& $R_h=0\Omega$)$S$理论值计算结果}
    \begin{tabular}{|c|c|c|c|c|c|}
      \hline
    $R_x$ & $R_{x1}$ & \multicolumn{3}{c|}{$R_{x2}$} & $R_{x3}$ \\
    \hline
    $R_1/R_2$ & 500/500 & 50/500 & 500/500 & 500/500(交换) & 500/500 \\
    \hline
    S     & $1.8 \times 10^3$ & $6.2 \times 10^2$ & $1.6 \times 10^3$ & $1.6 \times 10^3$ & $3.2\times 10^2$ \\
  \hline  
  \end{tabular}%
  \label{tab:addlabel}%
\end{table}%

% Table generated by Excel2LaTeX from sheet 'Sheet1'
\begin{table}[H]
  \centering
  \caption{$R_{x2}$不同测量条件$S$计算结果}
    \begin{tabular}{|c|c|c|}
      \hline
    $R_x$ & 条件    & S \\
    \hline
    \multirow{3}[0]{*}{$R_{x2}$} & E=2.0V \& $R_h=0(\Omega )$ \& $R_1/R_2=500/500$ & $8.0 \times 10^2$ \\
    \cline{2-3}
          & E=4.0V \& $R_h=0(\Omega )$ \& $R_1/R_2=500/5000$ & $3.2 \times 10^2$ \\
          \cline{2-3}
          & E=4.0V \& $R_h=3(k\Omega )$ \& $R_1/R_2=500/500$ & $2.2 \times 10^2$ \\
          
          \hline
    \end{tabular}%
  \label{tab:addlabel}%
\end{table}%

 下计算交换桥臂法测得的$R_{x2}$及其不确定度$\sigma_{x2}$:
 
 利用公式$$R=\sqrt{R_{01}\cdot R_{02}}$$
$$\sigma=\sqrt{(\frac{\partial R}{\partial R_{01}})^2\sigma_{R_{01}}^2+(\frac{\partial R}{\partial R_{02}})^2\sigma_{R_{02}}^2+(\delta R)^2}$$
$$(\frac{\partial R}{\partial R_{01}})^2\sigma_{R_{01}}^2=\frac{R_{02}}{4R_{01}}\cdot (\frac{0.1\% \times R_{01}}{\sqrt{3}})^2=0.011$$
$$(\frac{\partial R}{\partial R_{02}})^2\sigma_{R_{02}}^2=\frac{R_{01}}{4R_{02}}\cdot (\frac{0.1\% \times R_{02}}{\sqrt{3}})^2=0.011$$
$$(\delta R_x)^2=(\frac{0.2R_x}{S})^2=0.0026$$

得
$$R_{x2}=\sqrt{R_{01}\cdot R_{02}}=360.0(\Omega)$$
$$\sigma_{x2}=\sqrt{(\frac{\partial R}{\partial R_{01}})^2\sigma_{R_{01}}^2+(\frac{\partial R}{\partial R_{02}})^2\sigma_{R_{02}}^2+(\delta R)^2}=0.2(\Omega)$$
$$R_{x2}\pm\sigma_{x2}=(360.0\pm0.2)\Omega$$
\subsection{其余电阻测量不确定度}
其余电阻均未采用交换桥臂法。因此,其不确定度公式如下:$$\sigma=\sqrt{(\delta R)^2+(\frac{\partial R}{\partial R_{1}})^2\sigma_{R_{1}}^2+(\frac{\partial R}{\partial R_{2}})^2\sigma_{R_{2}}^2+(\frac{\partial R}{\partial R_{0}})^2\sigma_{R_{0}}^2}$$

又:
$$(\delta R)^2=(\frac{0.2R}{S})^2$$
$$(\frac{\partial R}{\partial R_{1}})^2\sigma_{R_{1}}^2=(\frac{R_0}{R_2})^2\frac{(0.1\% R_1)^2}{3}$$
$$(\frac{\partial R}{\partial R_{0}})^2\sigma_{R_{0}}^2=(\frac{R_1}{R_2})^2\frac{(0.1\% R_0)^2}{3}$$
$$(\frac{\partial R}{\partial R_{2}})^2\sigma_{R_{2}}^2=(\frac{R_1R_0}{R_2^2})^2\frac{(0.1\% R_2)^2}{3}$$

得计算结果对应表如下:
% Table generated by Excel2LaTeX from sheet 'Sheet1'
\begin{table}[htbp]
  \centering
  \caption{各测量电阻在给定条件下的不确定度计算值对应表}
  \resizebox{\textwidth}{!}
  {
    \begin{tabular}{|c|c|c|c|}
      \hline
      \diagbox[dir=NW]{实验} {值}{各项} & $R_x$ & 条件    & $\sigma(\Omega)$ \\
          \hline
    \multirow{3}[0]{*}{实验\Ronum{1}} & $R_{x1}$ & E=4.0V \& $R_h=0(\Omega )$ \& $R_1/R_2=500/500$ & 0.05\\
    \cline{2-4}
          & $R_{x2}$ & E=4.0V \& $R_h=0(\Omega )$ \& $R_1/R_2=50/500$ & 0.4 \\
          \cline{2-4}
          & $R_{x3}$ & E=4.0V \& $R_h=0(\Omega )$ \& $R_1/R_2=500/500$ & 5 \\
          \hline
    \multirow{3}[0]{*}{实验\Ronum{2}} & \multirow{3}[0]{*}{$R_{x2}$} & E=2.0V \& $R_h=0(\Omega )$ \& $R_1/R_2=500/500$ & 0.4 \\
    \cline{3-4}
          &       & E=4.0V \& $R_h=0(\Omega )$ \& $R_1/R_2=500/5000$ & 0.4 \\
          \cline{3-4}
          &       & E=4.0V \& $R_h=3(k\Omega )$ \& $R_1/R_2=500/500$ & 0.5 \\
          \hline
    \end{tabular}%
  }
  \label{tab:addlabel}%

\end{table}%
\subsection{S的计算值}
% Table generated by Excel2LaTeX from sheet 'Sheet1'
\begin{table}[H]
  \centering
  \caption{S的理论计算与实际计算值表}
  \resizebox{\textwidth}{!}
  {
    \begin{tabular}{|m{0.13\columnwidth}|c|c|c|c|}
      \hline
      \diagbox[dir=NW]{实验} {值}{各项} & $R_x$ & 条件    & $S_{\mbox{理论}}$ & $S_{\mbox{实际}}$ \\
      \hline
    \multirow{5}[0]{0.15\columnwidth}{实验\Ronum1} & $R_{x1}$ & E=4.0V \& $R_h=0(\Omega )$ \& $R_1/R_2=500/500$ & $1.8 \times 10^3$ & $1.9 \times 10^3$ \\
    \cline{2-5}
          & \multirow{3}[0]{*}{$R_{x2}$} & E=4.0V \& $R_h=0(\Omega )$ \& $R_1/R_2=50/500$ & $6.2 \times 10^2$ & $5.8 \times 10^2$ \\
          \cline{3-5}
          &       & E=4.0V \& $R_h=0(\Omega )$ \& $R_1/R_2=500/500$ & $1.6 \times 10^3$ & $1.4 \times 10^3$ \\
          \cline{3-5}
          &       & E=4.0V \& $R_h=0(\Omega )$ \& $R_1/R_2=500/500$ & $1.6 \times 10^3$ & $1.4 \times 10^3$ \\
          \cline{2-5}
          & $R_{x3}$ & E=4.0V \& $R_h=0(\Omega )$ \& $R_1/R_2=500/500$ & $3.2\times 10^2$ & $3.0\times 10^2$ \\
    \hline
          \multirow{3}[0]{0.15\columnwidth}{实验\Ronum2(略\Ronum1中相同条件)} & \multirow{3}[0]{*}{$R_{x2}$} & E=2.0V \& $R_h=0(\Omega )$ \& $R_1/R_2=500/500$ & $8.0 \times 10^2$ & $7.2 \times 10^2$ \\
          \cline{3-5}
          &       & E=4.0V \& $R_h=0(\Omega )$ \& $R_1/R_2=500/5000$ & $3.2 \times 10^2$ & $2.9 \times 10^2$ \\
          \cline{3-5}
          &       & E=4.0V \& $R_h=3(k\Omega )$ \& $R_1/R_2=500/500$ & $2.2 \times 10^2$ & $2.0 \times 10^2$ \\
    \hline
        \end{tabular}%
  }
  \label{tab:addlabel}%
\end{table}%
\section{思考题}
\paragraph{电源电压大幅下降}会。电源电压大幅下降使得灵敏度大幅度减小,使得读数误差增大。
\paragraph{电源电压稍有波动}不会。小幅度的波动对灵敏度影响较小。
\paragraph{在测量较低电阻时,导线电阻不可忽略}会。不可忽略的导线电阻不仅会体现在电阻的测量值内,还会影响电桥平衡过程。
\paragraph{检流计零点没有调准}会。最终的平衡位置电流很可能较大,使得系统误差变大。
\paragraph{检流计灵敏度较小}会。检流计灵敏度较小,会使得最终的读数误差较大。
\section{分析与讨论}
\subsection{分析各不确定度对总不确定度的贡献,并讨论如何提高精度}
\paragraph{不采用交换桥臂法}
在不采用交换桥臂法的情况下:
$$\sigma=\sqrt{(\delta R)^2+(\frac{\partial R}{\partial R_{1}})^2\sigma_{R_{1}}^2+(\frac{\partial R}{\partial R_{2}})^2\sigma_{R_{2}}^2+(\frac{\partial R}{\partial R_{0}})^2\sigma_{R_{0}}^2}$$

又:
$$(\delta R)^2=(\frac{0.2R}{S})^2=R^2\cdot \frac{0.04}{S^2}$$
$$(\frac{\partial R}{\partial R_{1}})^2\sigma_{R_{1}}^2=(\frac{R_0}{R_2})^2\frac{(0.1\% R_1)^2}{3}=R^2\cdot \frac{10^{-6}}{3}$$
$$(\frac{\partial R}{\partial R_{0}})^2\sigma_{R_{0}}^2=(\frac{R_1}{R_2})^2\frac{(0.1\% R_0)^2}{3}=R^2\cdot \frac{10^{-6}}{3}$$
$$(\frac{\partial R}{\partial R_{2}})^2\sigma_{R_{2}}^2=(\frac{R_1R_0}{R_2^2})^2\frac{(0.1\% R_2)^2}{3}=R^2\cdot \frac{10^{-6}}{3}$$

可见:各桥臂电阻的贡献相等(在所采用的阻值下相对误差相等),当读数误差大于各桥臂提供误差时,有:

$$\frac{0.04}{S^2}>\frac{10^{-6}}{3}\Leftrightarrow S<1.7\times 10^2 $$

但是本实验中诸数据$(S)$均不满足,故而在本实验中各桥臂电阻的贡献大于读数误差。

\paragraph{采用交换桥臂法}

根据公式:
$$\sigma=\sqrt{(\frac{\partial R}{\partial R_{01}})^2\sigma_{R_{01}}^2+(\frac{\partial R}{\partial R_{02}})^2\sigma_{R_{02}}^2+(\delta R)^2}$$
$$(\frac{\partial R}{\partial R_{01}})^2\sigma_{R_{01}}^2=\frac{R_{02}}{4R_{01}}\cdot (\frac{0.1\% \times R_{01}}{\sqrt{3}})^2=0.011$$
$$(\frac{\partial R}{\partial R_{02}})^2\sigma_{R_{02}}^2=\frac{R_{01}}{4R_{02}}\cdot (\frac{0.1\% \times R_{02}}{\sqrt{3}})^2=0.011$$
$$(\delta R_x)^2=(\frac{0.2R_x}{S})^2=0.0026$$

可见,读数产生的误差小于各桥臂提供的误差。

\paragraph{提高测量的精度}

首先,在一定范围内提高$S$有助于减少读数误差,同时,采用交换桥臂法,也可以减小误差,此外,还有选择合理的桥臂电阻等。

\subsection{灵敏度}
\paragraph{理论与实际}
由表6:S的理论计算与实际计算值表(见4页1.3),不难看出,S的理论值总是大于实际值,关于这点,我想有以下几个原因:
\subparagraph{1}电路中存在诸如接触电阻,导线电阻等阻值存在,而在理论计算中未考量。
\subparagraph{2}检流计中的阻尼等耗散一定能量,使振幅偏小。
\paragraph{依赖关系}
根据理论公式:$$S=\frac{S_GE}{R_1+R_2+R_3+R_4+(R_g+R_h)(2+\frac{R_1}{R_x}+\frac{R_0}{R_2})}$$

且满足关系$$R_x:R_1=R_1:R_2$$可得:

在给定检流计的基础上,$S$随$E的增加$、$R_h的减小$和$R_1/R_2比值趋于1$而增大。

\section{收获与感想}
电桥,一个简单的,却有着巨大实际用途的电路结构。

在我很小的时候,我就已经对电桥的巧妙有所耳闻。在中学时,也做了一些和电桥有关的实验。

每一次做电桥的实验,我对这个结构的理解也变得更深刻,从电桥的实验中,我们也能感受到一些重要的实验设计思想。

例如在惠斯通的年代,直接测量电阻很难,但是既然有检流计,又可以绕制标准电阻,就可以利用电桥将待测的电阻值改为标准电阻值,而这又是可测量的。

从中,我们可以认识到设计实验时,要考虑有些量由于实验条件限制直接测量不可行或误差较大,就要改变方法间接转化为误差较小的量来测算。

在今后的实验课程的学习中,我也会锻炼自己设计实验的能力。





\end{document} 