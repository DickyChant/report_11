\documentclass{ctexart}
    \usepackage{mathrsfs}
    \usepackage{multirow}
    \usepackage{graphicx}
    \usepackage{array}
    \usepackage{makecell}
    \usepackage{amsmath}
    \usepackage{booktabs}
    \usepackage{float}
    \usepackage{diagbox}
    \newcommand\mgape[1]{\gape{$\vcenter{\hbox{#1}}$}}
    \newcommand\Ronum[1]{\uppercase\expandafter{\romannumeral #1\relax}}
    \newcommand\ronum[1]{\romannumeral #1\relax}
    \author{钱思天\ 1600011388 No.8}
    \title{实验十九\ 分光计测量棱镜折射率 \ 实验报告}
    \begin{document}
      \maketitle
      \section{实验数据与处理}
      \subsection{实测数据}
      \subsubsection{顶角A测量}
      根据实测结果,并根据公式$$\phi=\frac{\theta_1-\theta_1'+\theta_2-\theta_2'}{2}$$
      % Table generated by Excel2LaTeX from sheet 'Sheet1'
\begin{table}[H]
    \centering
    \caption{顶角A测量结果}
      \begin{tabular}{|c|c|c|c|c|c|}
        \hline
      角度    & {$\theta_1$} & {$\theta_2$} & {$\theta_1'$} & {$\theta_2'$} & {$\phi$} \\
      \hline
      {1} & $167^\circ 18'$ & $107^\circ 22'$ & $347^\circ 16'$ & $287^\circ 16'$ & $59^\circ 58'$ \\
      \hline
      {2} & $167^\circ 20'$ & $107^\circ 20'$ & $347^\circ 17'$ & $287^\circ 18'$ & $60^\circ 00'$ \\
      \hline
      {3} & $167^\circ 18'$ & $107^\circ 22'$ & $347^\circ 20'$ & $287^\circ 18'$ & $59^\circ 59'$ \\
      \hline
      Average & $167^\circ 19'$ & $107^\circ 21'$ & $347^\circ 18'$ & $287^\circ 17'$ & $59^\circ 58.8'$ \\
      \hline
      \end{tabular}%
    \label{tab:addlabel}%
  \end{table}%
  得$$\mbox{顶角:}A=\bar{\phi}=59^\circ 58.8'$$
  \subsubsection{略入射法}
  根据实测结果,并根据公式$$\beta=\frac{\alpha_1-\alpha_1'+\alpha_2-\alpha_2'}{2}$$
  % Table generated by Excel2LaTeX from sheet 'Sheet1'
\begin{table}[H]
    \centering
    \caption{掠入射法测量结果}
      \begin{tabular}{|c|c|c|c|c|c|}
        \hline
      角度    & {$\alpha_1$} & {$\alpha_2$} & {$\alpha_1'$} & {$\alpha_2'$} & {$\beta$} \\
      \hline
      {1} & $150^\circ 31'$ & $109^\circ 07'$ & $330^\circ 28'$ & $289^\circ 04'$ & $41^\circ 24'$ \\
      \hline
      {2} & $150^\circ 31'$ & $109^\circ 09'$ & $330^\circ 29'$ & $289^\circ 04'$ & $41^\circ 24'$ \\
      \hline
      {3} & $150^\circ 31'$ & $109^\circ 06'$ & $330^\circ 27'$ & $289^\circ 04'$ & $41^\circ 24'$ \\
      \hline
      Average & $150^\circ 31'$ & $109^\circ 07'$ & $330^\circ 28'$ & $289^\circ 04'$ & $41^\circ 23.8'$ \\
    \hline  
    \end{tabular}%
    \label{tab:addlabel}%
  \end{table}%
  得$$\mbox{掠入射角:}\gamma=\bar{\beta}=41^\circ 23.8'$$
\subsubsection{最小偏转角法}
根据实测结果,并根据公式$$\eta=\frac{\zeta_1-\zeta_1'+\zeta_2-\zeta_2'}{2}$$
% Table generated by Excel2LaTeX from sheet 'Sheet1'
\begin{table}[H]
    \centering
    \caption{最小偏转角法测量结果}
      \begin{tabular}{|c|c|c|c|c|c|}
        \hline
      角度    & {$\zeta_1$} & {$\zeta_2$} & {$\zeta_1'$} & {$\zeta_2'$} & {$\eta$} \\
      \hline
      {1} & $97^\circ 52'$ & $43^\circ 43'$ & $277^\circ 52'$ & $223^\circ 46'$ & $54^\circ 08'$ \\
      \hline
      {2} & $97^\circ 52'$ & $43^\circ 46'$ & $277^\circ 48'$ & $223^\circ 45'$ & $54^\circ 05'$ \\
      \hline
      {3} & $97^\circ 53'$ & $43^\circ 45'$ & $277^\circ 50'$ & $223^\circ 43'$ & $54^\circ 08'$ \\
      \hline
      Average & $97^\circ 52'$ & $43^\circ 45'$ & $277^\circ 50'$ & $223^\circ 45'$ & $54^\circ 06.5'$ \\
      \hline
      \end{tabular}%
    \label{tab:addlabel}%
  \end{table}%
  得$$\mbox{最小偏转角:}\delta_m=\bar{\eta}=54^\circ 06.5'$$
  \subsection{计算}
  \subsubsection{顶角A的测量}
  根据顶角的计算公式,其不确定度分为两项:

  B类不确定度:分光计的允差$$e_0={1'}$$
  
  得$$\sigma_1=\frac{\frac{1}{3}\times(4e_0)}{\sqrt{3}}=\frac{4'}{3\sqrt{3}}=0.8'$$
  
  A类不确定度:根据标准差公式$$\sigma_2=\sqrt{\frac{\sum\limits_{i=1}^{3}{(A_i-\bar{A})^2}}{3\times2}}=0.4'$$

  故$$\sigma_A=\sqrt{\sigma_2^2+\sigma_1^2}=0.9'$$
  $$A\pm \sigma_A=59^\circ59.8'\pm0.9'$$
  同理,还可以求得掠入射法中的掠入射角$\gamma$和最小偏转角法中的最小偏转角$\delta_m$的不确定度
  $$\sigma_{\gamma}=0.8';\sigma_{\delta}=1.3'$$
  $$\gamma\pm\sigma_{\gamma}=41^\circ23.8'\pm 0.8'$$
  $$\delta_{m}\pm\sigma_{\delta}=54^\circ06.5'\pm 1.3'$$
  \subsubsection{掠入射法测折射率}
  由公式$$n=\sqrt{1+(\frac{\cos{A}+\sin{\gamma}}{\sin{A}})^2}=1.6732$$
  $$\sigma_n=\sqrt{(\frac{\partial n}{\partial A})^2\sigma_A^2+(\frac{\partial n}{\partial \gamma})^2\sigma_{\gamma}^2}$$
  
  而$$(\frac{\partial n}{\partial A})^2\sigma_A^2=\sigma_A^2(\frac{-2 \csc (A) (\cos (A)+\sin (\gamma ))-2 \cot (A) \csc ^2(A) (\cos (A)+\sin (\gamma ))^2}{2 \sqrt{\csc ^2(A) (\cos (A)+\sin (\gamma ))^2+1}})^2$$
  $$(\frac{\partial n}{\partial \gamma})^2\sigma_{\gamma}^2=\sigma_{\gamma}^2(\frac{\csc ^2(A) \cos (\gamma ) (\cos (A)+\sin (\gamma ))}{\sqrt{\csc ^2(A) (\cos (A)+\sin (\gamma ))^2+1}})^2$$
  为计算不确定度,将角度制转化为弧度制计算,得$$\sigma_n=\sqrt{(\frac{\partial n}{\partial A})^2\sigma_A^2+(\frac{\partial n}{\partial \gamma})^2\sigma_{\gamma}^2}=0.0004$$
  $$n\pm \sigma_n=1.6732\pm 0.0004$$
  \subsubsection{最小偏转角法测折射率}
  由公式$$n=\csc \left(\frac{A}{2}\right) \sin \left(\frac{1}{2} \left(A+\delta _m\right)\right)=1.6787$$
  $$\sigma_n=\sqrt{(\frac{\partial n}{\partial A})^2\sigma_A^2+(\frac{\partial n}{\partial \delta_m})^2\sigma_{\delta_m}^2}$$
  
  而$$(\frac{\partial n}{\partial A})^2\sigma_A^2=\sigma_A^2(-\frac{1}{2} \csc ^2\left(\frac{A}{2}\right) \sin \left(\frac{\delta _m}{2}\right))^2$$
  $$(\frac{\partial n}{\partial \delta_m})^2\sigma_{\delta_m}^2=\sigma_{\delta_m}^2(\frac{1}{2} \csc \left(\frac{A}{2}\right) \cos \left(\frac{1}{2} \left(A+\delta _m\right)\right))^2$$
  为计算不确定度,将角度制转化为弧度制计算,得$$\sigma_n=\sqrt{(\frac{\partial n}{\partial A})^2\sigma_A^2+(\frac{\partial n}{\partial \delta_m})^2\sigma_{\delta_m}^2}=0.0003$$
  $$n\pm \sigma_n=1.6787\pm 0.0003$$
  \subsection{选做部分-测量汞灯的其余谱线}
  根据实测结果,并根据公式$$\psi=\frac{\xi_1-\xi_1'+\xi_2-\xi_2'}{2}$$
  $$n=\csc \left(\frac{A}{2}\right) \sin \left(\frac{1}{2} \left(A+\psi\right)\right)$$
  
 % Table generated by Excel2LaTeX from sheet 'Sheet1'
\begin{table}[htbp]
    \centering
    \caption{测量及计算结果}
      \begin{tabular}{|c|c|c|c|c|c|c|}
        \hline
      角度    & {$\xi_1$} & {$\xi_2$} & {$\xi_1'$} & {$\xi_2'$} & {$\psi$} & {$n$} \\
      \hline
      黄     & $98^\circ 10'$ & $43^\circ 42'$ & $278^\circ 11'$ & $223^\circ 42'$ & $54^\circ 49'$ & 1.6853 \\
      \hline
      暗绿    & $97^\circ 30'$ & $43^\circ 41'$ & $277^\circ 26'$ & $223^\circ 44'$ & $53^\circ 66'$ & 1.6786 \\
      \hline
      紫     & $99^\circ 30'$ & $43^\circ 44'$ & $279^\circ 21'$ & $223^\circ 45'$ & $55^\circ 61'$ & 1.6966 \\
      \hline
      暗紫    & $98^\circ 50'$ & $43^\circ 45'$ & $278^\circ 50'$ & $223^\circ 43'$ & $55^\circ 06'$ & 1.6880 \\
      \hline
      \end{tabular}%
    \label{tab:addlabel}%
  \end{table}%
  
  
\section{分析与讨论}
\paragraph{答}在我看来,本次实验的误差来源有以下几点
\subparagraph{1}狭缝的宽度,过窄的宽度亮度较小,过宽的又存在像的尺度问题,都会对实验产生影响
\subparagraph{2}不可避免的仪器允差等
\section{收获与感想}
分光计,在高中时我就对它又爱又恨。爱的是分光计的使用很有趣而原理又很巧妙,精度也很高;恨的,就是他那有些繁琐的调节了。

这次做实验也不例外,一大半的时间都消耗在了分光计的调节上。

当然,除了调节之外,这次实验的收获还是很大的。

其一就是分光计的使用了,每一次使用分光计,都不由得为它的精巧所折服。
而且从分光计的原理中,我还感受到了对于几何关系的运用。

此外,调节分光计时,我也对了逐步逼近,控制自由度等思想有了更深的理解。



\end{document} 